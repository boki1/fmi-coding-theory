\documentclass[11pt, oneside]{article}   	% use "amsart" instead of "article" for AMSLaTeX format

% Language setting
\usepackage[utf8]{inputenc}
\usepackage[bulgarian]{babel}


\usepackage{geometry}                		% See geometry.pdf to learn the layout options. There are lots.
\geometry{letterpaper}                   		% ... or a4paper or a5paper or ...
\usepackage[parfill]{parskip}    			% Activate to begin paragraphs with an empty line rather than an indent
\usepackage{graphicx}				% Use pdf, png, jpg, or eps§ with pdflatex; use eps in DVI mode
								% TeX will automatically convert eps --> pdf in pdflatex
\usepackage{amssymb}
\usepackage{mathtools}
\usepackage{enumerate}
\usepackage{multicol}
\usepackage{tikz}
\usepackage{amsmath}

\usetikzlibrary{arrows}

\def\firstcircle{(90:1.75cm) circle (2.5cm)}
\def\secondcircle{(210:1.75cm) circle (2.5cm)}
\def\thirdcircle{(330:1.75cm) circle (2.5cm)}

\setcounter{MaxMatrixCols}{40}

%SetFonts

%SetFonts


\title{Домашна работа 1 \\ {\Large Увод в теория на кодирането}}
\author{Кристиян Стоименов \\ {ф.н 3MI0400121, ФМИ}}
\date{Април 2023}

\begin{document}

\maketitle

\section*{Задача 1}
Да се съставят матрици на Адамар от ред 12 и ред 16 и да се обясни начина,
по който са получени. От тях да се съставят оптималните нелинейни кодове с дължини 10,
11 и 12 и 14, 15 и 16, съответно.

\textbf{Решение:}
За да съставим Адамарова матрица от ред 12, използваме метода на Пейли. Търсим квадратичните остатъци $(mod\ 11)$ - то ни дава общо 12 класа остатъци. Необходимо е да проверим само числата в интервала $[1, \frac{p-1}{2}], p = 11$. Получаваме следните стойности:
\begin{multicols}{2}
\begin{itemize}
    \item $1^{2} = 1 < 11$
    \item $2^{2} = 4 < 11$
    \item $3^{2} = 9 < 11$
    \item $4^{2}\equiv 5 \pmod{11}$
    \item $5^{2}\equiv 3 \pmod{11}$
\end{itemize}
\end{multicols}

Следователно числата $1, 3, 4, 5, 9$ са \textit{квадратични остатъци}, а $2, 6, 7, 8, 10$ - \textit{неостатъци}. 0 не считаме нито за остатък, нито за неостатък. Използвайки следната дефиницията на характеристична функция, \[\chi(i)=\begin{cases*}
1& когато и е квадратичен остатък по $(mod\ p)$,\\
-1& когато не е квадратичен остатък по $(mod\ p)$,\\
0& когато $p\ |\ i$
\end{cases*} \] пресмятаме $\chi(1) = \chi(3) = \chi(4) = \chi(5) = \chi(9) = 1$ и $\chi(2) = \chi(6) = \chi(7) = \chi(8) = \chi(10) = -1$. След това съставяме матрица $Q_{11\times11} = (q_{ij}), q_{ij} = \chi(j - i)$:

\begin{center}
$Q_{11} = $\begin{pmatrix}
0 & 1 & -1 & 1 & 1 & 1 & -1 & -1 & -1 & 1 & -1 \\
-1 & 0 & 1 & -1 & 1 & 1 & 1 & -1 & -1 & -1 & 1 \\
1 & -1 & 0 & 1 & -1 & 1 & 1 & 1 & -1 & -1 & -1 \\
-1 & 1 & -1 & 0 & 1 & -1 & 1 & 1 & 1 & -1 & -1 \\
-1 & -1 & 1 & -1 & 0 & 1 & -1 & 1 & 1 & 1 & -1 \\
-1 & -1 & -1 & 1 & -1 & 0 & 1 & -1 & 1 & 1 & 1 \\
1 & -1 & -1 & -1 & 1 & -1 & 0 & 1 & -1 & 1 & 1 \\
1 & 1 & -1 & -1 & -1 & 1 & -1 & 0 & 1 & -1 & 1 \\
1 & 1 & 1 & -1 & -1 & -1 & 1 & -1 & 0 & 1 & -1 \\
-1 & 1 & 1 & 1 & -1 & -1 & -1 & 1 & -1 & 0 & 1 \\
1 & -1 & 1 & 1 & 1 & -1 & -1 & -1 & 1 & -1 & 0
\end{pmatrix}
\end{center}

Тогава получаваме Адамарова матрица от ред 12 като вградим получената $Q_{11}$ по следния начин:

\begin{center}
$A_{12} = $\begin{pmatrix}
1 & 1 & \cdots & 1  \\
1 &   $Q_{11} - E$ & \empty   \\
\vdots             \\ 
1 & \empty & \empty
\end{pmatrix}
$ = $\begin{pmatrix}
-1 & 1 & -1 & 1 & 1 & 1 & -1 & -1 & -1 & 1 & -1 \\
-1 & -1 & 1 & -1 & 1 & 1 & 1 & -1 & -1 & -1 & 1 \\
1 & -1 & -1 & 1 & -1 & 1 & 1 & 1 & -1 & -1 & -1 \\
-1 & 1 & -1 & -1 & 1 & -1 & 1 & 1 & 1 & -1 & -1 \\
-1 & -1 & 1 & -1 & -1 & 1 & -1 & 1 & 1 & 1 & -1 \\
-1 & -1 & -1 & 1 & -1 & -1 & 1 & -1 & 1 & 1 & 1 \\
1 & -1 & -1 & -1 & 1 & -1 & -1 & 1 & -1 & 1 & 1 \\
1 & 1 & -1 & -1 & -1 & 1 & -1 & -1 & 1 & -1 & 1 \\
1 & 1 & 1 & -1 & -1 & -1 & 1 & -1 & -1 & 1 & -1 \\
-1 & 1 & 1 & 1 & -1 & -1 & -1 & 1 & -1 & -1 & 1 \\
1 & -1 & 1 & 1 & 1 & -1 & -1 & -1 & 1 & -1 & -1
\end{pmatrix}
\end{center}

\newline

За да съставим Адамарова матрица от ред 16, използваме метода на Силвестър. Знаем, че $A_{1} = (1)$ е Адамарова и поради това, можем да конструираме $A_{2} = $\begin{pmatrix}
$A_{1}$ & $A_{1}$ \\
$A_{1}$ & $-A_{1}$
\end{pmatrix} $ = $ \begin{pmatrix}
1 & 1 \\
1 & -1
\end{pmatrix}. Аналогично създаваме $A_{4}$, $A_{8}$ и $A_{16}$:

$A_{4} = $\begin{pmatrix}
$A_{2}$ & $A_{2}$ \\
$A_{2}$ & $-A_{2}$
\end{pmatrix}$ = $\begin{pmatrix}
1 & 1 & 1 & 1 \\
1 & -1 & 1 & -1 \\
1 & 1 & -1 & -1 \\
1 & -1 & -1 & 1
\end{pmatrix}

$A_{8} = $\begin{pmatrix}
$A_{4}$ & $A_{4}$ \\
$A_{4}$ & $-A_{4}$
\end{pmatrix}$ = $\begin{pmatrix}
1 & 1  & 1  & 1  & 1 & 1  & 1  & 1 \\
1 & -1 & 1  & -1 & 1 & -1 & 1  & -1 \\
1 & 1  & -1 & -1 & 1 & 1  & -1 & -1 \\
1 & -1 & -1 & 1  & 1 & -1 & -1 & 1 \\
1 & 1  & 1  & 1  & -1 & -1 & -1 & -1 \\
1 & -1 & 1  & -1 & -1 & 1 & -1 & 1 \\
1 & 1  & -1 & -1 & -1 & -1 & 1 & 1 \\
1 & -1 & -1 & 1  & -1 & 1 & 1 & -1
\end{pmatrix}

$A_{16} = $\begin{pmatrix}
$A_{8}$ & $A_{8}$ \\
$A_{8}$ & $-A_{8}$
\end{pmatrix}$ = $\begin{pmatrix}
1 & 1  & 1  & 1  & 1 & 1  & 1  & 1   & 1 & 1  & 1  & 1  & 1 & 1  & 1  & 1 \\ 
1 & -1 & 1  & -1 & 1 & -1 & 1  & -1  & 1 & -1 & 1  & -1 & 1 & -1 & 1  & -1 \\
1 & 1  & -1 & -1 & 1 & 1  & -1 & -1  & 1 & 1  & -1 & -1 & 1 & 1  & -1 & -1 \\
1 & -1 & -1 & 1  & 1 & -1 & -1 & 1   & 1 & -1 & -1 & 1  & 1 & -1 & -1 & 1 \\
1 & 1  & 1  & 1  & -1 & -1 & -1 & -1 & 1 & 1  & 1  & 1  & -1 & -1 & -1 & -1 \\
1 & -1 & 1  & -1 & -1 & 1 & -1 & 1   & 1 & -1 & 1  & -1 & -1 & 1 & -1 & 1 \\
1 & 1  & -1 & -1 & -1 & -1 & 1 & 1   & 1 & 1  & -1 & -1 & -1 & -1 & 1 & 1 \\
1 & -1 & -1 & 1  & -1 & 1 & 1 & -1   & 1 & -1 & -1 & 1  & -1 & 1 & 1 & -1 \\
1 & 1  & 1  & 1  & 1 & 1  & 1  & 1   & -1 & -1  & -1  & -1  & -1 & -1  & -1  & -1 \\ 
1 & -1 & 1  & -1 & 1 & -1 & 1  & -1  & -1 & 1   & -1  & 1   & -1 & 1 & -1  & 1 \\
1 & 1  & -1 & -1 & 1 & 1  & -1 & -1  & -1 & -1  & 1   & 1   & -1 & -1  & 1 & 1 \\
1 & -1 & -1 & 1  & 1 & -1 & -1 & 1   & -1 & 1   & 1   & -1  & -1 & 1 & 1 & -1 \\
1 & 1  & 1  & 1  & -1 & -1 & -1 & -1 & -1 & -1  & -1  & -1  & 1 & 1 & 1 & 1 \\
1 & -1 & 1  & -1 & -1 & 1 & -1 & 1   & -1 & 1   & -1  & 1   & 1 & -1 & 1 & -1 \\
1 & 1  & -1 & -1 & -1 & -1 & 1 & 1   & -1 & -1  & 1   & 1   & 1 & 1 & -1 & -1
\end{pmatrix}

\end{document}
