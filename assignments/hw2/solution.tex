\documentclass[11pt, oneside]{article}   	% use "amsart" instead of "article" for AMSLaTeX format
\usepackage[mathscr]{euscript}

\usepackage[utf8]{inputenc}
\usepackage[bulgarian]{babel}
\usepackage[parfill]{parskip}
\usepackage{graphicx}
\usepackage{mathtools}
\usepackage{enumerate}
\usepackage{multicol}
\usepackage{tikz}
\usepackage{amsmath}
\usepackage{amsfonts}
\usepackage{amssymb}
\usepackage{calligra}
\usepackage{calrsfs}
\usepackage{booktabs}
\usepackage[left=2cm,right=2cm,top=2cm,bottom=2cm]{geometry}
\geometry{letterpaper}

\usetikzlibrary{arrows}

\def\firstcircle{(90:1.75cm) circle (2.5cm)}
\def\secondcircle{(210:1.75cm) circle (2.5cm)}
\def\thirdcircle{(330:1.75cm) circle (2.5cm)}

\setcounter{MaxMatrixCols}{40}

\DeclareMathAlphabet{\mathcalligra}{T1}{calligra}{m}{n}

\title{Домашна работа 2 \\ {\Large Увод в теория на кодирането}}
\author{Кристиян Стоименов \\ {ф.н 3MI0400121, ФМИ}}
\date{Май 2023}

\begin{document}

\maketitle

\section*{Задача 1}
Да се напише векторът, който съответства на булевата функция $f = 1 + v_3 + v_1v_2$
\begin{itemize}
    \item ако $f$ е функция на три променливи
    \item ако $f$ е функция на четири променливи
\end{itemize}

\textbf{Решение:}

\textit{Ако $f$ е функция на три промеливи:}

Откриваме стойностите за $f$ като заместим $v_1$, $v_2$ и $v_3$ в полинома, вземайки стойности от всеки един от редовете. Решението е написано на Таблица \ref{tab:problem_1_f_as_fun_of_3}.

\begin{table}[htbp]
    \centering
    \begin{tabular}{ccccc}
        \toprule
            \textbf{Стойност} &\textbf{$v_1$} & \textbf{$v_2$} & \textbf{$v_3$} & \textbf{$f$} \\
        \midrule
            0 & 0 & 0 & 0 & 1 \\
            1 & 0 & 0 & 1 & 0 \\
            2 & 0 & 1 & 0 & 1 \\
            3 & 0 & 1 & 1 & 0 \\
            4 & 1 & 0 & 0 & 1 \\
            5 & 1 & 0 & 1 & 0 \\
            6 & 1 & 1 & 0 & 0 \\
            7 & 1 & 1 & 1 & 1 \\
        \bottomrule
    \end{tabular}
    \caption{\textit{Стойности на $f(v_1, v_2, v_3) = 1 + v_3 + v_1v_2$}}
    \label{tab:problem_1_f_as_fun_of_3}
\end{table}


\textit{Ако $f$ е функция на четири промеливи:}

Тъй като $v_4$ не участва в полинома, то неговата стойност няма отражение върху функционалната стойност. Поради тази причина, няма нужда отново да смятаме $f(v_1, v_2, v_3, *)$. Решението е написано на Таблица \ref{tab:problem_1_f_as_fun_of_4}.

\begin{table}[htbp]
    \begin{minipage}[t]{.50\textwidth}
        \begin{tabular}{cccccc}
            \toprule
            \textbf{Стойност} &\textbf{$v_1$} & \textbf{$v_2$} & \textbf{$v_3$} & \textbf{$v_4$} & \textbf{$f$}\\
            \midrule
                 0 & 0 & 0 & 0 & 0 & 1 \\
                 1 & 0 & 0 & 0 & 1 & 1 \\
                 2 & 0 & 0 & 1 & 0 & 0 \\
                 3 & 0 & 0 & 1 & 1 & 0 \\
                 4 & 0 & 1 & 0 & 0 & 1 \\
                 5 & 0 & 1 & 0 & 1 & 1 \\
                 6 & 0 & 1 & 1 & 0 & 0 \\
                 7 & 0 & 1 & 1 & 1 & 0 \\
            \bottomrule
        \end{tabular}
    \end{minipage} \hfill
    \begin{minipage}[t]{.50\textwidth}
        \begin{tabular}{cccccc}
            \toprule
                \textbf{Стойност} &\textbf{$x_1$} & \textbf{$x_2$} & \textbf{$x_3$} & \textbf{$x_4$} & \textbf{$f$} \\
            \midrule
                 8 & 1 & 0 & 0 & 0 & 1 \\
                 9 & 1 & 0 & 0 & 1 & 1 \\
                10 & 1 & 0 & 0 & 0 & 0 \\
                11 & 1 & 0 & 1 & 1 & 0 \\
                12 & 1 & 1 & 1 & 0 & 0 \\
                13 & 1 & 1 & 0 & 1 & 0 \\
                14 & 1 & 1 & 1 & 0 & 1 \\
                15 & 1 & 1 & 1 & 1 & 1 \\
            \bottomrule
        \end{tabular}
    \end{minipage} \hfill
    \caption{\textit{Стойности на $f(v_1, v_2, v_3, v_4) = 1 + v_3 + v_1v_2$}}
    \label{tab:problem_1_f_as_fun_of_4}
\end{table}

\section*{Задача 2}
\label{sec:problem_2}
Да се намери полинома на Жегалкин, съответстващ на $f$, зададена със Таблица \ref{tab:problem_2_truth_table}.

\begin{table}[htbp]
    \begin{minipage}[t]{.50\textwidth}
        \begin{tabular}{cccccc}
            \toprule
                \textbf{Стойност} &\textbf{$x_1$} & \textbf{$x_2$} & \textbf{$x_3$} & \textbf{$x_4$} & \textbf{$f$} \\
            \midrule
                 0 & 0 & 0 & 0 & 0 & 1 \\
                 1 & 0 & 0 & 0 & 1 & 1 \\
                 2 & 0 & 0 & 1 & 0 & 0 \\
                 3 & 0 & 0 & 1 & 1 & 1 \\
                 4 & 0 & 1 & 0 & 0 & 0 \\
                 5 & 0 & 1 & 0 & 1 & 0 \\
                 6 & 0 & 1 & 1 & 0 & 0 \\
                 7 & 0 & 1 & 1 & 1 & 0 \\
            \bottomrule
        \end{tabular}
    \end{minipage} \hfill
    \begin{minipage}[t]{.50\textwidth}
        \begin{tabular}{cccccc}
            \toprule
                \textbf{Стойност} &\textbf{$x_1$} & \textbf{$x_2$} & \textbf{$x_3$} & \textbf{$x_4$} & \textbf{$f$} \\
            \midrule
                 8 & 1 & 0 & 0 & 0 & 1 \\
                 9 & 1 & 0 & 0 & 1 & 0 \\
                10 & 1 & 0 & 1 & 0 & 1 \\
                11 & 1 & 0 & 1 & 1 & 1 \\
                12 & 1 & 1 & 0 & 0 & 0 \\
                13 & 1 & 1 & 0 & 1 & 1 \\
                14 & 1 & 1 & 1 & 0 & 1 \\
                15 & 1 & 1 & 1 & 1 & 1 \\
            \bottomrule
        \end{tabular}
    \end{minipage} \hfill
    \caption{\textit{Функцията $f$, чието представяне чрез полином се търси в Задача \ref{sec:problem_2}.}}
    \label{tab:problem_2_truth_table}
\end{table}

\textbf{Решение:}

Представяме функцията $f$ като функция с един по-малко аргумент и други две функции $f_0$ и $f_1$. Извършваме аналгични действия, докато не стигнем до функция на една променлива, чиято стойност можем да определим непострествено - това са функциите $f_{***}$.

\begin{multicols}{2}
\begin{align*}
    f &= (1 + x_1)f_0 + x_1f_1 \\
    f_0 &= (1 + x_2)f_{00} + x_2f_{01} \\
    f_1 &= (1 + x_2)f_{10} + x_2f_{11} \\
    f_{00} &= (1 + x_3)f_{000} + x_3f_{001} \\
    f_{01} &= (1 + x_3)f_{010} + x_3f_{011} \\
    f_{10} &= (1 + x_3)f_{100} + x_3f_{101} \\
    f_{11} &= (1 + x_3)f_{110} + x_3f_{111} \\
\end{align*}
\columnbreak
\begin{align*}
    f_{000} &= 1 \\
    f_{001} &= x_4 \\
    f_{010} &= 0 \\
    f_{011} &= 0 \\
    f_{100} &= 1 + x_4 \\
    f_{101} &= 1 \\
    f_{110} &= x_4 \\
    f_{111} &= 1 \\
\end{align*}
\end{multicols}

Сега "се връщаме нагоре", за да получим полином за функцията $f$.

\begin{align*}
    f_{00} &= (1 + x_3)f_{000} + x_3f_{001} = (1 + x_3) + x_3x_4 = 1 + x_3x_4 + x_3 \\
    f_{01} &= (1 + x_3)f_{010} + x_3f_{011} = (1 + x_3).0 + x_3.0 = 0 \\
    f_{10} &= (1 + x_3)f_{100} + x_3f_{101} = (1 + x_3)(1 + x_4) + x_3 = 1 + x_3x_4 + x_4 \\
    f_{11} &= (1 + x_3)f_{110} + x_3f_{111} = (1 + x_3)x_4 + x_3 = x_3x_4 + x_3 + x_4 \\
\end{align*}
\vspace{-7ex}
\begin{align*}
    f_0 &= (1 + x_2)f_{00} + x_2f_{01} = (1 + x_2)(1 + x_3x_4 + x_3) + x_2.0 = 1 + x_3x_4 + x_3 + x_2 + x_2x_3x_4 + x_2x_3 \\
    f_1 &= (1 + x_2)f_{10} + x_2f_{11} = (1 + x_2)(1 + x_3x_4 + x_4) + x_2(x_3x_4 + x_3 + x_4) = 1 + x_3x_4 + x_2x_3 + x_4 + x_2 \\
\end{align*}
\vspace{-7ex}
\begin{align*}
    f &= (1 + x_1)f_0 + x_1f_1 = (1 + x_1)(1 + x_3x_4 + x_3 + x_2 + x_2x_3x_4 + x_2x_3) + x_1(1 + x_2x_3 + x_3x_4 + x_2 + x_4) = \\
    &= 1 + x_1x_2x_3x_4 + x_2x_3x_4 + x_1x_4 + x_1x_3 + x_2x_3 + x_3x_4 + x_2 + x_3 \\
\end{align*}

\section*{Задача 3}
Да се опишат параметрите на всички $RM$ кодове на пет промеливи и да се посочи количеството грешки, които могат да поправят.

\textbf{Решение:}

Щом разглеждаме код на пет промеливи, то той ще съдържа всички двоични вектори с дължина $2^5 = 32$ (т. е $m = 5$). Също така най-високата възможна степен за едночлен, и респективно за полиноми, е 5. Тогава трябва да изследваме $RM(r, 5)$, където $0 \leq r \leq 5$.

Дължината на кодовите думи е фиксирана: $2^5$. При зададено $r$ от посочения интервал размерността $k$ пресмятаме като
\[
    k = \sum_{i=0}^{r} \binom{m}{i}
\]
За минималното разстояние знаем, че се получава като $d = 2^{m - r}$. Количеството грешки, които открива коът е $ \leq 2^{m-r} - 1$, а количеството на тези, които поправя - \[
t \leq \lfloor \frac{2^{m-r}-1}{2} \rfloor
\]

Обобщено получаваме

\begin{table}[htbp]
    \centering
    \begin{tabular}{ccc}
        \toprule
            \textbf{Код} & \textbf{Параметри} & \textbf{Граница за $t$} \\
        \midrule
            $ RM(0, 5) $ & $ [32, 1, 32] $ & $ t \leq 15 $ \\
            $ RM(1, 5) $ & $ [32, 6, 16] $ & $ t \leq 7 $ \\
            $ RM(2, 5) $ & $ [32, 16, 8] $ & $ t \leq 3 $ \\
            $ RM(3, 5) $ & $ [32, 26, 4] $ & $ t \leq 1 $ \\
            $ RM(4, 5) $ & $ [32, 31, 2] $ & $ t \leq 0 $ \\
            $ RM(5, 5) $ & $ [32, 32, 1] $ & $ t \leq 0 $ \\
        \bottomrule
    \end{tabular}
    \caption{\textit{Описание на кодове $RM(r, 5)$, където $0 \leq r \leq 5$}}.
\end{table}

\section*{Задача 4}
Да се напише пораждащата матрица на кода $RM(1, 4)$ и да се декодират следните вектори чрез декодера на Рид:
\begin{itemize}
  \item $y = 1000 0101 1110 0101$
  \item $z = 1001 0110 0110 1001$
  \item $t = 1110 1110 1110 1110$
\end{itemize}

\textbf{Решение:}

Размерността на кода е
\begin{align*}
    k &= \sum_{i=0}^{1} \binom{4}{i} = \binom{4}{0} + \binom{4}{1} = 5,
\end{align*}
а дължината на всеки вектор от него е \[2^{4} = 16.\]
В пораждащата матрица трябва да включим двоичните вектои, съответстващи на едночлени от степени 0 и 1 ($<= r$, където $r = 1$).

\begin{table}[htbp]
    \begin{minipage}[t]{.50\textwidth}
        \textit{Степен 0}
        \begin{tabular}{ccccc}
            \toprule
                \textbf{$x_1$} & \textbf{$x_2$} & \textbf{$x_3$} & \textbf{$x_4$} & \textbf{$f$} \\
            \midrule
                * & * & * & * & 1 \\
            \bottomrule
        \end{tabular}
    \end{minipage} \hfill
    \begin{minipage}[t]{.50\textwidth}
        \textit{Степен 1}
        \begin{tabular}{ccccc}
            \toprule
                \textbf{$x_1$} & \textbf{$x_2$} & \textbf{$x_3$} & \textbf{$x_4$} & \textbf{$f = x_1$} \\
            \midrule
                0 & * & * & * & 0 \\
                1 & * & * & * & 1 \\
            \bottomrule
        \end{tabular}
    \end{minipage} \hfill
\end{table}

Аналогично прилагаме и за $f = x_2$, $f = x_3$ и $f = x_4$, за да получим още четири вектора. Накрая, получените вектори подреждаме в матрица, за да получим пораждащата:

\[
\text{$G = $}
\underbrace{
    \begin{pmatrix}
        1 & 1 & 1 & 1 & 1 & 1 & 1 & 1 & 1 & 1 & 1 & 1 & 1 & 1 & 1 & 1 \\
        0 & 0 & 0 & 0 & 0 & 0 & 0 & 0 & 1 & 1 & 1 & 1 & 1 & 1 & 1 & 1 \\
        0 & 0 & 0 & 0 & 1 & 1 & 1 & 1 & 0 & 0 & 0 & 0 & 1 & 1 & 1 & 1 \\
        0 & 0 & 1 & 1 & 0 & 0 & 1 & 1 & 0 & 0 & 1 & 1 & 0 & 0 & 1 & 1 \\
        0 & 1 & 0 & 1 & 0 & 1 & 0 & 1 & 0 & 1 & 0 & 1 & 0 & 1 & 0 & 1 \\
    \end{pmatrix}
}_{\text{$5 \times 16$}}
\]

Декодираме получена дума $y$, търсейки изпратения информационен вектор $a = (a_1, a_2, a_3, a_4, a_5)$. Връзката между двете е чрез пораждащата матрица:

\begin{alignat*}{6}
y = a.G = (&a_1, &&a_1 + a_5, &&a_1 + a_4, &&a_1 + a_4 + a_5, \\
          &a_1 + a_3, &&a_1 + a_3 + a_5, &&a_1 + a_3 + a_4, &&a_1 + a_3 + a_4 + a_5, \\
          &a_1 + a_2, &&a_1 + a_2 + a_5, &&a_1 + a_2 + a_4, &&a_1 + a_2 + a_4 + a_5, \\
          &a_1 + a_2 + a_3, \hspace{10pt} &&a_1 + a_2 + a_3 + a_5, \hspace{10pt} &&a_1 + a_2 + a_3 + a_4, \hspace{10pt} &&a_1 + a_2 + a_3 + a_4 + a_5)
\end{alignat*}

Ако полученият вектор $y = (y_{1}, y_{2}, \hdots, y_{16})$, то можем да получим следните равенства:

\begin{align*}
y_1 &= a_1 \\
\end{align*}

\begin{align*}
y_9 &= a_1 + a_2 = y_1 + a_2 \Rightarrow a_2 = y_1 + y_9 \\
y_{10} &= a_1 + a_2 + a_5 = a_2 + (a_1 + a_5) = y_2 + a_2 \Rightarrow a_2 = y_2 + y_{10} \\
y_{11} &= a_1 + a_2 + a_4 = a_2 + (a_1 + a_4) = a_2 + y_3 \Rightarrow a_2 = y_3 + y_{11} \\
y_{12} &= a_1 + a_2 + a_4 + a_5 = a_2 + (a_1 + a_4 + a_5) = a_2 + y_4 \Rightarrow a_2 = y_4 + y_{12} \\
y_{13} &= a_1 + a_2 + a_3 = a_2 + (a_1 + a_3) = a_2 + y_5 \Rightarrow a_2 = y_5 + y_{13} \\
y_{14} &= a_1 + a_2 + a_3 + a_5 = a_2 + (a_1 + a_3 + a_5) = a_2 + y_6 \Rightarrow a_2 = y_6 + y_{14} \\
y_{15} &= a_1 + a_2 + a_3 + a_4 = a_2 + (a_1 + a_3 + a_4) = a_2 + y_7 \Rightarrow a_2 = y_7 + y_{15} \\
y_{16} &= a_1 + a_2 + a_3 + a_4 + a_5 = a_2 + (a_1 + a_3 + a_4 + a_5) = a_2 + y_8 \Rightarrow a_2 = y_8 + y_{16} \\
\end{align*}
\vspace{-7ex}
\begin{align*}
y_5 &= a_3 + a_1 = a_3 + y_1 \Rightarrow a_3 = y_1 + y_5 \\
y_6 &= a_3 + (a_1 + a_5) = a_3 + y_2 \Rightarrow a_3 = y_2 + y_6 \\
y_7 &= a_3 + (a_1 + a_4) = a_3 + y_3 \Rightarrow a_3 = y_3 + y_7 \\
y_8 &= a_3 + (a_1 + a_3 + a_5) = a_3 + y_4 \Rightarrow a_3 = y_4 + y_8 \\
y_{13} &= a_3 + (a_1 + a_2) = a_3 + y_9 \Rightarrow a_3 = y_9 + y_{13} \\
y_{14} &= a_3 + (a_1 + a_2 + a_5) = a_3 + y_{10} \Rightarrow a_3 = y_{10} + y_{14} \\
y_{15} &= a_3 + (a_1 + a_2 + a_4) = a_3 + y_{11} \Rightarrow a_3 = y_{11} + y_{15} \\
y_{16} &= a_3 + (a_1 + a_2 + a_4 + a_5) = a_3 + y_{12} \Rightarrow a_3 = y_{12} + y_{16} \\
\end{align*}
\vspace{-7ex}
\begin{align*}
y_3 &= a_4 + a_1 = a_4 + y_1 \Rightarrow a_4 = y_1 + y_3 \\
y_4 &= a_4 + (a_1 + a_5) = a_4 + y_2 \Rightarrow a_4 = y_2 + y_4 \\
y_7 &= a_4 + (a_1 + a_3) = a_4 + y_5 \Rightarrow a_4 = y_5 + y_7 \\
y_8 &= a_4 + (a_1 + a_3 + a_5) = a_4 + y_6 \Rightarrow a_4 = y_6 + y_8 \\
y_{11} &= a_4 + (a_1 + a_2) = a_4 + y_9 \Rightarrow a_4 = y_9 + y_{11} \\
y_{12} &= a_4 + (a_1 + a_2 + a_5) = a_4 + y_{10} \Rightarrow a_4 = y_{10} + y_{11} \\
y_{15} &= a_4 + (a_1 + a_2 + a_3) = a_4 + y_{13} \Rightarrow a_4 = y_{13} + y_{15} \\
y_{16} &= a_4 + (a_1 + a_2 + a_3 + a_5) = a_4 + y_{14} \Rightarrow a_4 = y_{14} + y_{16} \\
\end{align*}
\vspace{-7ex}
\begin{align*}
y_2 &= a_5 + a_1 = a_5 + y_1 \Rightarrow a_5 = y_1 + y_2 \\
y_4 &= a_5 + (a_1 + a_4) = a_5 + y_3 \Rightarrow a_5 = y_3 + y_4 \\
y_6 &= a_5 + (a_1 + a_3) = a_5 + y_5 \Rightarrow a_5 = y_5 + y_6 \\
y_8 &= a_5 + (a_1 + a_3 + a_4) = a_5 + y_7 \Rightarrow a_5 = y_7 + y_8 \\
y_{10} &= a_5 + (a_1 + a_2) = a_5 + y_9 \Rightarrow a_5 = y_9 + y_{10} \\
y_{12} &= a_5 + (a_1 + a_2 + a_4) = a_5 + y_{11} \Rightarrow a_5 = y_{11} + y_{12} \\
y_{14} &= a_5 + (a_1 + a_2 + a_3) = a_5 + y_{13} \Rightarrow a_5 = y_{13} + y_{14} \\
y_{16} &= a_5 + (a_1 + a_2 + a_3 + a_4) = a_5 + y_{15} \Rightarrow a_5 = y_{15} + y_{16} \\
\end{align*}

\textit{Обобщено:}
\begin{align*}
a_1 &= y_1 \\
a_2 &= y_1 + y_9 = y_2 + y_{10} = y_3 + y_{11} = y_4 + y_{12} = y_5 + y_{13} = y_6 + y_{14} = y_7 + y_{15} = y_8 + y_{16} \\
a_3 &= y_1 + y_5 = y_2 + y_6 = y_3 + y_7 = y_4 + y_8 = y_9 + y_{13} = y_{10} + y_{14} = y_{11} + y_{15} = y_{12} + y_{16} \\
a_4 &= y_1 + y_3 = y_2 + y_4 = y_5 + y_7 = y_6 + y_8 = y_9 + y_{11} = y_{10} + y_{12} = y_{13} + y_{15} = y_{14} + y_{16} \\
a_5 &= y_1 + y_2 = y_3 + y_4 = y_5 + y_6 = y_7 + y_8 = y_9 + y_{10} = y_{11} + y_{12} = y_{13} + y_{14} = y_{15} + y_{16} \\
\end{align*}

Нека получената кодова дума за декодиране е $y = (1000 0101 1110 0101)$. Тогава пресмятаме всички възможни стойности за всеки един от битовете на информационния вектор, за да получим

\begin{table}[htbp]
    \centering
    \begin{tabular}{cccc}
        \toprule
            \textbf{Бит} & \textbf{Брой на "0"} & \textbf{Брой на "1"} & \textbf{Стойност} \\
        \midrule
            $ a_1 $ & - & - & 1 \\
            $ a_2 $ & 6 & 2 & 0 \\
            $ a_3 $ & 2 & 6 & 1 \\
            $ a_4 $ & 6 & 2 & 0 \\
            $ a_5 $ & 2 & 6 & 1 \\
        \bottomrule
    \end{tabular}
    \caption{\textit{Декодиране на $y = (1000 0101 1110 0101)$}}
\end{table}

Окончателно получаваме, че думата $y$ се декодира до $10101$.

Нека сега получената кодова дума за декодиране е $z = (1001 0110 0110 1001)$. Тогава пресмятаме по сходен начин

\begin{table}[htbp]
    \centering
    \begin{tabular}{cccc}
        \toprule
            \textbf{Бит} & \textbf{Брой на "0"} & \textbf{Брой на "1"} & \textbf{Стойност} \\
        \midrule
            $ a_1 $ & - & - & 1 \\
            $ a_2 $ & 0 & 8 & 1 \\
            $ a_3 $ & 0 & 8 & 1 \\
            $ a_4 $ & 0 & 8 & 0 \\
            $ a_5 $ & 0 & 8 & 1 \\
        \bottomrule
    \end{tabular}
    \caption{\textit{Декодиране на $z = (1001 0110 0110 1001)$}}
\end{table}

Окончателно получаваме, че думата $z$ се декодира до $11111$.

Нека накрая получената кодова дума да бъде $t = (1110 1110 1110 1110)$. Тогава пресмятаме по сходен начин

\begin{table}[htbp]
    \centering
    \begin{tabular}{cccc}
        \toprule
            \textbf{Бит} & \textbf{Брой на "0"} & \textbf{Брой на "1"} & \textbf{Стойност} \\
        \midrule
            $ a_1 $ & - & - & 1 \\
            $ a_2 $ & 8 & 0 & 0 \\
            $ a_3 $ & 8 & 0 & 0 \\
            $ a_4 $ & 4 & 4 & - \\
            $ a_5 $ & 4 & 4 & - \\
        \bottomrule
    \end{tabular}
    \caption{\textit{Декодиране на $t = (1110 1110 1110 1110)$}}
\end{table}

Така можем да заключим, че $t$ не се декодира недвусмислено.

\section*{Задача 5}
Да се декодират следните вектори чрез постъпково мажоритарно декодиране, приложено върху $RM(1, 3)$ код:
\begin{itemize}
  \item $y = 1011 1010$
  \item $z = 1001 1001$
  \item $t = 0111 0111$
\end{itemize}

\textbf{Решение:}

При $RM(1, 3)$ се интересуваме от точките на афинната геометрия $AG(3, 2)$, които са
$P_0 = (000)$, $P_1 = (001)$, $P_2 = (010)$, $P_3 = (011)$, $P_4 = (100)$, $P_5 = (101)$, $P_6 = (110)$, $P_7 = (111)$.

Нека разглеждаме получените вектори като характеристични вектори на подмножество от точки от афинната геометрия $AG(3, 2)$.
\begin{align*}
M_{y} &= \{P_0,P_2,P_3,P_4,P_6\} \subset AG(3, 2), \\
M_{z} &= \{P_0,P_3,P_4,P_7\} \subset AG(3, 2), \\
M_{t} &= \{P_1,P_2,P_3,P_5,P_6,P_7\} \subset AG(3, 2)
\end{align*}

Разглеждаме всички равнини (двумерни подпространства на $AG(3, 2)$) и тяхната четност при пресичане с тези подмножества от точки $M$ като ги подреждаме в таблица.

Разглеждаме всички прави, зададени чрез две от точките $P$ и техните четности, определени от четностите на равнините, в които се срещат. Представяме ги сходно на това равнините и техните четности.

Накрая разглеждаме и четностите на точките в съответните подмножества.

За изпратени кодови думи получаваме съответно
\begin{align*}
a_{y} &= (), \\
a_{z} &= (), \\
a_{t} &= ().
\end{align*}

\textit{Споменатите таблици се намират на следващата страница.}

\begin{table}
\begin{minipage}[t]{.3\textwidth}
\begin{tabular}{cccc}
\toprule
\textbf{Равнина} & \textbf{$M_{y}$} & \textbf{$M_{z}$} & \textbf{$M_{t}$}\\
\midrule
$ P_{0}, P_{1}, P_{2}, P_{3} $ & 1 & 0 & 1 \\
$ P_{0}, P_{1}, P_{2}, P_{4} $ & 1 & 0 & 0 \\
$ P_{0}, P_{1}, P_{2}, P_{5} $ & 0 & 1 & 1 \\
$ P_{0}, P_{1}, P_{2}, P_{6} $ & 1 & 1 & 1 \\
$ P_{0}, P_{1}, P_{2}, P_{7} $ & 0 & 0 & 1 \\
$ P_{0}, P_{1}, P_{3}, P_{4} $ & 1 & 1 & 0 \\
$ P_{0}, P_{1}, P_{3}, P_{5} $ & 0 & 0 & 1 \\
$ P_{0}, P_{1}, P_{3}, P_{6} $ & 1 & 0 & 1 \\
$ P_{0}, P_{1}, P_{3}, P_{7} $ & 0 & 1 & 1 \\
$ P_{0}, P_{1}, P_{4}, P_{5} $ & 0 & 0 & 0 \\
$ P_{0}, P_{1}, P_{4}, P_{6} $ & 1 & 0 & 0 \\
$ P_{0}, P_{1}, P_{4}, P_{7} $ & 0 & 1 & 0 \\
$ P_{0}, P_{1}, P_{5}, P_{6} $ & 0 & 1 & 1 \\
$ P_{0}, P_{1}, P_{5}, P_{7} $ & 1 & 0 & 1 \\
$ P_{0}, P_{1}, P_{6}, P_{7} $ & 0 & 0 & 1 \\
$ P_{0}, P_{2}, P_{3}, P_{4} $ & 0 & 1 & 0 \\
$ P_{0}, P_{2}, P_{3}, P_{5} $ & 1 & 0 & 1 \\
$ P_{0}, P_{2}, P_{3}, P_{6} $ & 0 & 0 & 1 \\
$ P_{0}, P_{2}, P_{3}, P_{7} $ & 1 & 1 & 1 \\
$ P_{0}, P_{2}, P_{4}, P_{5} $ & 1 & 0 & 0 \\
$ P_{0}, P_{2}, P_{4}, P_{6} $ & 0 & 0 & 0 \\
$ P_{0}, P_{2}, P_{4}, P_{7} $ & 1 & 1 & 0 \\
$ P_{0}, P_{2}, P_{5}, P_{6} $ & 1 & 1 & 1 \\
$ P_{0}, P_{2}, P_{5}, P_{7} $ & 0 & 0 & 1 \\
$ P_{0}, P_{2}, P_{6}, P_{7} $ & 1 & 0 & 1 \\
$ P_{0}, P_{3}, P_{4}, P_{5} $ & 1 & 1 & 0 \\
$ P_{0}, P_{3}, P_{4}, P_{6} $ & 0 & 1 & 0 \\
$ P_{0}, P_{3}, P_{4}, P_{7} $ & 1 & 0 & 0 \\
$ P_{0}, P_{3}, P_{5}, P_{6} $ & 1 & 0 & 1 \\
$ P_{0}, P_{3}, P_{5}, P_{7} $ & 0 & 1 & 1 \\
$ P_{0}, P_{3}, P_{6}, P_{7} $ & 1 & 1 & 1 \\
$ P_{0}, P_{4}, P_{5}, P_{6} $ & 1 & 0 & 0 \\
$ P_{0}, P_{4}, P_{5}, P_{7} $ & 0 & 1 & 0 \\
$ P_{0}, P_{4}, P_{6}, P_{7} $ & 1 & 1 & 0 \\
$ P_{0}, P_{5}, P_{6}, P_{7} $ & 0 & 0 & 1 \\
\bottomrule
\end{tabular}
\end{minipage} \hfill
\begin{minipage}[t]{.3\textwidth}
\begin{tabular}{cccc}
\toprule
\textbf{Равнина} & \textbf{$M_{y}$} & \textbf{$M_{z}$} & \textbf{$M_{t}$}\\
\midrule
$ P_{1}, P_{2}, P_{3}, P_{4} $ & 1 & 0 & 1 \\
$ P_{1}, P_{2}, P_{3}, P_{5} $ & 0 & 1 & 0 \\
$ P_{1}, P_{2}, P_{3}, P_{6} $ & 1 & 1 & 0 \\
$ P_{1}, P_{2}, P_{3}, P_{7} $ & 0 & 0 & 0 \\
$ P_{1}, P_{2}, P_{4}, P_{5} $ & 0 & 1 & 1 \\
$ P_{1}, P_{2}, P_{4}, P_{6} $ & 1 & 1 & 1 \\
$ P_{1}, P_{2}, P_{4}, P_{7} $ & 0 & 0 & 1 \\
$ P_{1}, P_{2}, P_{5}, P_{6} $ & 0 & 0 & 0 \\
$ P_{1}, P_{2}, P_{5}, P_{7} $ & 1 & 1 & 0 \\
$ P_{1}, P_{2}, P_{6}, P_{7} $ & 0 & 1 & 0 \\
$ P_{1}, P_{3}, P_{4}, P_{5} $ & 0 & 0 & 1 \\
$ P_{1}, P_{3}, P_{4}, P_{6} $ & 1 & 0 & 1 \\
$ P_{1}, P_{3}, P_{4}, P_{7} $ & 0 & 1 & 1 \\
$ P_{1}, P_{3}, P_{5}, P_{6} $ & 0 & 1 & 0 \\
$ P_{1}, P_{3}, P_{5}, P_{7} $ & 1 & 0 & 0 \\
$ P_{1}, P_{3}, P_{6}, P_{7} $ & 0 & 0 & 0 \\
$ P_{1}, P_{4}, P_{5}, P_{6} $ & 0 & 1 & 1 \\
$ P_{1}, P_{4}, P_{5}, P_{7} $ & 1 & 0 & 1 \\
$ P_{1}, P_{4}, P_{6}, P_{7} $ & 0 & 0 & 1 \\
$ P_{1}, P_{5}, P_{6}, P_{7} $ & 1 & 1 & 0 \\
$ P_{2}, P_{3}, P_{4}, P_{5} $ & 1 & 0 & 1 \\
$ P_{2}, P_{3}, P_{4}, P_{6} $ & 0 & 0 & 1 \\
$ P_{2}, P_{3}, P_{4}, P_{7} $ & 1 & 1 & 1 \\
$ P_{2}, P_{3}, P_{5}, P_{6} $ & 1 & 1 & 0 \\
$ P_{2}, P_{3}, P_{5}, P_{7} $ & 0 & 0 & 0 \\
$ P_{2}, P_{3}, P_{6}, P_{7} $ & 1 & 0 & 0 \\
$ P_{2}, P_{4}, P_{5}, P_{6} $ & 1 & 1 & 1 \\
$ P_{2}, P_{4}, P_{5}, P_{7} $ & 0 & 0 & 1 \\
$ P_{2}, P_{4}, P_{6}, P_{7} $ & 1 & 0 & 1 \\
$ P_{2}, P_{5}, P_{6}, P_{7} $ & 0 & 1 & 0 \\
$ P_{3}, P_{4}, P_{5}, P_{6} $ & 1 & 0 & 1 \\
$ P_{3}, P_{4}, P_{5}, P_{7} $ & 0 & 1 & 1 \\
$ P_{3}, P_{4}, P_{6}, P_{7} $ & 1 & 1 & 1 \\
$ P_{3}, P_{5}, P_{6}, P_{7} $ & 0 & 0 & 0 \\
$ P_{4}, P_{5}, P_{6}, P_{7} $ & 0 & 0 & 1 \\
\bottomrule
\end{tabular}
\end{minipage} \hfill
\begin{minipage}[t]{.3\textwidth}
\begin{tabular}{cccc}
\toprule
\textbf{Права} & \textbf{$M_{y}$} & \textbf{$M_{z}$} & \textbf{$M_{t}$}\\
\midrule
$ P_{0}, P_{1} $ & 0 & 0 & 1 \\
$ P_{0}, P_{2} $ & 1 & 0 & 1 \\
$ P_{0}, P_{3} $ & 1 & 1 & 1 \\
$ P_{0}, P_{4} $ & 1 & 1 & 0 \\
$ P_{0}, P_{5} $ & 0 & 0 & 1 \\
$ P_{0}, P_{6} $ & 1 & 0 & 1 \\
$ P_{0}, P_{7} $ & 0 & 1 & 1 \\
$ P_{1}, P_{2} $ & 0 & 1 & 1 \\
$ P_{1}, P_{3} $ & 0 & 0 & 1 \\
$ P_{1}, P_{4} $ & 0 & 0 & 1 \\
$ P_{1}, P_{5} $ & 0 & 1 & 1 \\
$ P_{1}, P_{6} $ & 0 & 1 & 1 \\
$ P_{1}, P_{7} $ & 0 & 0 & 1 \\
$ P_{2}, P_{3} $ & 1 & 0 & 1 \\
$ P_{2}, P_{4} $ & 1 & 0 & 1 \\
$ P_{2}, P_{5} $ & 0 & 1 & 1 \\
$ P_{2}, P_{6} $ & 1 & 1 & 1 \\
$ P_{2}, P_{7} $ & 0 & 0 & 1 \\
$ P_{3}, P_{4} $ & 1 & 1 & 1 \\
$ P_{3}, P_{5} $ & 0 & 0 & 1 \\
$ P_{3}, P_{6} $ & 1 & 0 & 1 \\
$ P_{3}, P_{7} $ & 0 & 1 & 1 \\
$ P_{4}, P_{5} $ & 0 & 0 & 1 \\
$ P_{4}, P_{6} $ & 1 & 0 & 1 \\
$ P_{4}, P_{7} $ & 0 & 1 & 1 \\
$ P_{5}, P_{6} $ & 0 & 1 & 1 \\
$ P_{5}, P_{7} $ & 0 & 0 & 1 \\
$ P_{6}, P_{7} $ & 0 & 0 & 1 \\
\bottomrule
\toprule
\textbf{Точка} & \textbf{$M_{y}$} & \textbf{$M_{z}$} & \textbf{$M_{t}$}\\
\midrule
$ P_{0} $ & 0 & 0 & 1 \\
$ P_{1} $ & 1 & 0 & 1 \\
$ P_{2} $ & 1 & 1 & 1 \\
$ P_{3} $ & 1 & 1 & 0 \\
$ P_{4} $ & 0 & 0 & 1 \\
$ P_{5} $ & 1 & 0 & 1 \\
$ P_{6} $ & 0 & 1 & 1 \\
$ P_{7} $ & 0 & 1 & 1 \\
\bottomrule
\end{tabular}
\end{minipage} \hfill
\caption{\textit{Подпространсва на $AG(3, 2)$ и техните четности спрямо множествата от точки зададени от подадените кодови вектори.}}
\end{table}

\end{document}
